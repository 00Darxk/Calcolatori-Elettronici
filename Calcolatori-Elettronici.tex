\documentclass{article}

\usepackage{cancel}
%\usepackage{tikz}
%\usepackage{circuitikz}
\usepackage{amsmath}
\usepackage[includehead,nomarginpar]{geometry}
\usepackage{graphicx}
\usepackage{amsfonts} 
\usepackage{verbatim}
\usepackage{mathrsfs}  
\usepackage{lmodern}
\usepackage{braket}
\usepackage{bookmark}
\usepackage{fancyhdr}
\usepackage{romanbarpagenumber}
%\usepackage{minted}
%\usepackage{subfig}
\usepackage[italian]{babel}
%\usepackage{float}
\allowdisplaybreaks

\setlength{\headheight}{12.0pt}
\addtolength{\topmargin}{-12.0pt}

\hypersetup{
    colorlinks=true,
    linkcolor=black,
}

\makeatother

\numberwithin{equation}{subsection}

\fancypagestyle{link}{\fancyhf{}\renewcommand{\headrulewidth}{0pt}\fancyfoot[C]{Sorgente del file LaTeX disponibile al seguente link: \url{https://github.com/00Darxk/Calcolatori-Elettronici}}}

\begin{document}

\title{%
    \textbf{Calcolatori Elettronici}  \\ 
    \large Appunti delle Lezioni di Calcolatori Elettronici \\
    \textit{Anno Accademico: 2023/24}}
\author{\textit{Giacomo Sturm}}
\date{\textit{Dipartimento di Ingegneria Civile, Informatica e delle Tecnologie Aeronautiche \\
Università degli Studi ``Roma Tre"}}

\maketitle
\thispagestyle{link}

\clearpage


\pagestyle{fancy}
\fancyhead{}\fancyfoot{}
\fancyhead[C]{\textit{Calcolatori Elettronici - Università degli Studi ``Roma Tre"}}
\fancyfoot[C]{\thepage}
\pagenumbering{Roman}

\tableofcontents

\clearpage
\pagenumbering{arabic}


% prof:
%% big data
%% LLM
%% Tirocini (Azienda/Laboratori) 
%% Incubare Startup

% scopi del corso
%% 6 CFU
%% rec 80-85% programma anni precedenti
%% cenni di assembly alla fine del corso
%% simulatori e casi di studio


%% programma diviso in due parti: (da aggiungere a README)
% I parte:
%% storia e topologia dei calcolatori: calcolatori moderni; x86, ARM, AVL;
%% sistemi di numerazione binaria: rappresentazione floating point numbers, standard IEEE 754;
%% ppt...
% II parte:
%% bus: ppt...
%% micro-architettura di una cpu: ppt...
%% programmazione assembly x86


% modalità d'esame (ppt):


%\section{Introduzione}

\section{Storia e Tipologia dei Calcolatori}

%% analisi degli aspetti hardware 
% funzionamento di un microprocessore, facendo riferimento a processori moderni e reali. 

\subsection{Evoluzione delle Architetture}

%% ppt


Un calcolatore è un oggetto che fornisce un risultato, dato un insieme dei dati inseriti. 


L'evoluzione delle architetture dei controllori elettronici si è svolta principalmente negli ultimi settant'anni. Il processo complessivo che ha portato alla nascita 
dei calcolatori moderni viene divisa in generazioni. 
Si chiama generazione zero, l'insieme di calcolatori analogici, progettati per risolvere semplici operazioni, ideati da Pascal e Leibniz. 

Il primo calcolatore programmabile venne ideato da Charles Babbage. Costruì prima una macchina differenziale in grado di calcolare funzioni polinomiali, mentre progettò la 
prima macchina programmabile, completamente analogica, in grado di leggere un input scritto su piastre di rame, e fornire un output, sempre su piastre di rame, utilizzando 
i dati e le operazioni inserite in input. Per poter operare su questi dati di input per fornire istruzioni alla macchina è necessario un linguaggio di programmazione, e la 
prima persona che ha tentato di implementare il linguaggio di Babbage fu Ada. %cognome 


Il passaggio seguente in questa evoluzione venne trainato principalmente dai fondi bellici per realizzare macchine elettromeccaniche. La prima generazione si indica il 
periodo dove vennero realizzati calcolatori abbandonando componenti meccanici. La prima macchina del genere venne create da Alan Turing per decifrare il codice Enigma, 
realizzata tramite valvole, chiamata Colossus. 



Dopo la guerra non servirono più a scopi bellici, per cui si tentò di vendere calcolatori sul mercato, creando la prima società di sviluppo e vendita di calcolatori 
sul mercato, Eniac. 

Negli anni '50 John von Neumann descrisse l'idea di un calcolatore moderno, dove i dati vengono memorizzati su degli indirizzi di memoria. 

L'IBM cominciò la sua storia vendendo calcolatori nel 1953, e continuo ad essere rilevante in questo ambito fino agli anni '80. 
In queste macchine ogni elemento viene definito dal termine ``word'', composto da un certo numero di bit. 


La fine degli anni '50 e l'inizio degli anni '60 vide l'avvento dei transistor, utilizzati in questi anni per la creazione di calcolatori basati su transistor, la società 
rivale della IBM che venne creata in questo periodo fu la DEC. Utilizzando transistor vennero diminuiti i costi, e venne introdotta l'idea di utilizzare uno schermo grafico 
per interagire con l'utente. Il primo calcolatore costruito dalla DEC, PDP-1, fu il primo calcolatore di massa. 

Per accedere ad un calcolatore si utilizzavano dei terminali, tramite un canale di comunicazione chiamato bus, per permettere anche la comunicazione tra elementi interni al 
calcolatore prodotti tra società differenti. 
Si parla comunque di grossi calcolatori per applicazioni scientifiche, militari o di pubblica amministrazione, chiamati ``mainframe'', dove ogni utente accedeva al calcolatore 
tramite terminali. 

Fino agli anni '80 non vennero introdotti cambiamenti radicali all'architettura dei calcolatori, invece i miglioramenti di questi periodi ai calcolatori riguardarono soprattutto 
l'ottimizzazione del software e dell'hardware. 

I primi ``Personal Computer'' vennero introdotti negli anni '80, dall'IBM, che fornì pubblicamente l'architettura del calcolatore. In seguito aumentò in enorme maniera 
l'utilizzo di PC, trainato dall'aumento delle capacità della CPU, e dalla diminuzione dei costi delle memorie principali e secondarie. 


%% ppt... 

% V Computer Invisibili
La maggior parte dei dispositivi moderni contengono microcontrollori, piccoli processori, distribuiti in un contesto completamente pervasivo, su ogni dispositivo collegato 
ad una qualche fonte di energia. Questo concetto viene chiamato anche dell'``Internet of Things''.  

\subsection{Legge di Moore}

Uno dei fondatori dell'Intel, Moore, negli anni '60 definì empiricamente l'omonima legge, secondo cui il numero di transistor su un chip, CPU, memoria, etc., raddoppia ogni 18 
mesi. Questo corrisponde ad un aumento del 60\% all'anno. 

L'evoluzione reale sembra aver seguito l'andamento descritto da Moore, ma recentemente l'evoluzione sta rallentando, a causa dei limiti fisici nella 
realizzazione dei transistor. Per cui esiste un limite superiore al numero di transistor su un unico chip. 
Per misurare la quantità di transistor su un singolo chip si utilizza la grandezza ``Livello di Integrazione'', si riescono a creare chip con un livello di 
integrazione nell'ordine di grandezza dei nanometri, ma livelli di integrazione superiore sono difficilmente realizzabili. Il limite teorico per memorizzare un bit di informazione 
corrisponde allo spin di un elettrone, per cui ci si aspetta una riduzione in questo andamento nei prossimi anni. 

Oltre alla legge di Moore son presenti diverse statistiche per misurare l'evoluzione tecnologica dei processori. Dagli anni 2000 si utilizzano più di un core su un unico 
processo, introducendo semplice forme di parallelismo. Uno dei motivi principali per cui vennero introdotte queste architetture deriva dal limite alla frequenza di funzionamento 
di un processore, poiché all'aumentare della frequenza aumenta il calore prodotto da un processore. Le frequenze maggiori raggiunte da processori si trovano nell'ordine dei 
GHz, queste producono calore fino a 100 Watt. Per aumentare le prestazioni senza aumentare la frequenza, si introducono quindi forme di parallelismo nei processori. Una forma 
semplice consiste nella duplicazione dei componenti, oppure della ``pipeline'', che realizza le stesse prestazioni senza introdurre parallelismo fisico. 
Gia dal 2000 quindi la frequenza operazionale dei processori è rimasta costante, ed ha subito una leggera diminuzione, allo stesso modo del calore generato da un singolo 
processore. Anche se è possibile realizzare processori mono-core che lavorano a frequenze molto elevate, il costo associato al raffreddamento dei componenti non lo rende un 
approccio economicamente attuabile. 

Dal punto di vista tecnologico per migliorare le prestazioni, bisogna cercare forme diverse di realizzazione di processori, che non utilizzano transistor, una di queste possibili 
tecnologie riguardano la computazione quantistica. 


La legge di Nathan afferma che il software è un gas, riempie sempre completamente qualsiasi contenitore in cui viene inserito. Per cui molto velocemente e facilmente un 
calcolatore diventa obsoleto, questo alimenta un circolo vizioso che spinge l'evoluzione tecnologica, e rappresenta quindi la legge di Moore. 

\subsection{Tipologie di Processori}

Un calcolatore è un dispositivo in grado di ricevere dei dati, di memorizzare in piccola parte i dati, di elaborare i dati, e di produrre un output. In generale un qualsiasi d
dispositivo elettronico in grado di soddisfare queste quattro specifiche può essere considerato un calcolatore. 
Si possono quindi definire diverse classi di calcolatori o processori sulla base delle loro prestazioni, all'aumentare delle prestazioni aumenta quindi il costo associato ad 
un dato processore. Esistono calcolatori monouso o ``usa e getta'', e microprocessori di basso costo, utilizzati negli elettrodomestici, automobili, o altri oggetti che non richiedono di capacità 
di computazione elevate, e soddisfano compiti specifici. Processori più evoluti, ma sempre specializzati, vengono utilizzai per applicazioni ``mobile'', oppure per piattaforme 
di gioco. L'unica differenza rispetto ad un Personal Computer è la loro specializzazione, mentre i processori di questa categoria svolgono applicazioni più generali 
``General Purpose'', in grado di essere programmati. Processori ancora più avanzati vengono utilizzati per fornire servizi, non per l'elaborazione personale, e vengono chiamati 
server, ma in termini di tecnologia non presenta differenze evidenti rispetto ad un PC. Veniva utilizzati processori ancora più potenti, chiamati ``Mainframe'', sulla base della 
centralizzazione della computazione, dove un singolo processore soddisfa le richieste di tutti gli utenti, ma non vengono più utilizzati a favore dell'elaborazione 
distribuita. 


Processori usa e getta come gli RFID ``Radio Frequency IDentification'' rappresentano la categoria di processori più diffusa, sono tipicamente passivi, senza batteria, ma esistono 
dispositivi attivi, di dimensione 
molto contenuta, nell'ordine di qualche millimetro, contenente un piccolo processore dotati di un transponder radio. Contiene una memoria di 128 bit complessivi. Il transponder 
è in grado di ricevere segnali su una certa frequenza, inviato da un lettore, questo segnale radio fornisce ulteriormente l'energia necessaria per alimentare il processore 
che invia il numero memorizzato in memoria. 
Gli RFID attivi dotati di una batteria non necessitano di essere molto vicini al lettore per operare, uno di questi dispositivi è il ``Telepass''. 


Microprocessori sono oggetti di plastica che contengono un processore, una piccola memoria, e forniscono un collegamento con l'esterno da vari piedini metallici. Necessitano di 
un'alimentazione esterna, su uno di questi piedini. Questi processori non sono programmabili, e vengono usati in applicazioni di controllo. 


Processori specializzati, non estendibili, ma di prestazioni molto superiori ai microprocessori sono i ``Game Computer'', che presentano effetti grafici speciali, per cui 
in generale presentano un processore grafico specializzato ``Graphical Processing Unit'' o GPU, ed un software di base limitato. Oltre alla memoria di base chiamata RAM, contengono la memoria di video, per gestire 
la visualizzazione a schermo chiamata VRAM. Generalmente questi processori CPU o GPU lavorano a non più di 4 GHz, per fornire informazioni sulle prestazioni di un processore 
si considera la banda di un processore, che rappresenta il numero di operazioni effettuabili in un dato intervallo di tempo. Si usa l'unità di misura FLOPS ``FLOating points Per Second'' 
supponendo il caso peggiore, quindi operazioni su numeri a virgola mobile. In generale questi dispositivi presentano una banda nell'ordine dei tera FLOPS. 
Questi sistemi sono chiusi, quindi non è possibile aumentare le prestazioni aggiungendo ulteriori chip al dispositivo. 
Appartengono alla stessa categoria le applicazioni Mobile, che presentano processori anche a otto core, con frequenze inferiori, poiché non presentano un sistema di raffreddamento 
attivo, e contengono una batteria e non un'alimentazione costante, per cui si utilizzano queste frequenze per diminuire il consumo energetico del processore. I processori 
utilizzati nell'ambito Mobile appartengono alla famiglia ARM, questa non è una casa produttrice come Intel o AMD, ma rappresentano una categoria di processori che vengono 
realizzati da diversi produttori, poiché è un'architettura aperta, di cui sono note le specifiche ed il linguaggio macchina. Questo modello di mercato si basa interamente sulle 
licenze vendute dalla casa produttrice ARM, per cui si quando si parla di un processore di questa famiglia, si include anche la casa produttrice che ha prodotto il processore. 
Recentemente la Apple ha esteso l'uso di processori ARM anche su applicazioni di Personal Computer. Su questi dispositivi le funzionalità I/O vengono fornite tramite un 
'interfaccia grafica basate su touch-screen. 


Il Personal Computer si riferisce alla disciplina dell'elaborazione personale dei dati. Sono processori di specifiche non molto diverse dalle precedenti, ma sono 
programmabili. Tutti questi dispositivi sono connessi alla rete, per cui appartengono all'Internet of Things. La differenza tra un PC ed un server è la disciplina secondo cui 
l'elaborazione dei dati non è personale, ma fornisce un servizio. Tipicamente questi servizi vengono fornite tramite diversi server che lavorano in parallelo secondo la 
disciplina COW ``Cluster Of Workstation'', collegati 
tramite una rete ad alta velocità, che presentano una ridondanza nella replicazione dei dati, in caso uno di questi server abbia un malfunzionamento. La tendenza al parallelismo 
è quindi presente non solo a livello microscopico sui singolo processori, ma anche a livello macroscopico utilizzando più server. Permettono di continuare ad erogare il servizio in 
caso di un malfunzionamento, ed è molto raro che la maggior parte dei server nel cluster falliscono contemporaneamente. La realizzazione di questi server segue la disciplina 
della scalabilità orizzontale, ovvero vengono aggiunti nuovi server all'aumentare degli utenti, quando invece è presente un unico server si parla di scalabilità verticale, dove 
per fornire servizi a più utenti si aumentano le prestazioni di un unico server. 


Esiste una tendenza di molte organizzazione a non realizzare un sistema di elaborazione con risorse proprie, ma utilizzare risorse nel Cloud, un'esempio molto diffuso è l'AWS, 
o gli Amazon Web Service, che forniscono memoria, memorizzazione di dati, e capacità di computazione accessibile nel Cloud. In questo approccio si sta ritornando all'approccio 
del Mainframe, ma in questo caso il terminale di accesso alle risorse fornite nel Cloud è anch'esso un calcolatore avente risorse di calcolo proprie, per elaborare una 
parte dei dati ``In Premise''. 



Verranno trattati tre diversi tipi di processori, appartenenti alla famiglia Intel e ARM, ed un processore appartenente alla famiglia dei microcontrollori della famiglia AVR.
%%\subsection{Architettura Intel}
%% ppt

Il primo processore commercializzato dalla Intel è il 4004, con una frequenze tra le frazioni di un MHz, ed in grado di gestire poche centinaia di bit di memoria. Una variante 
di questo processore, specializzato per microcontrollori il 8008, fornì la base per la creazione del primo processore ``general purpose'' su un circuito integrato, il 8080.  

La prima cifra nel nome corrisponde al tipo di architettura del processore, indica il numero di bit in cui vengono salvati e gestiti i dati sui registri del processore. 
A partire dagli anni '90 si cominciò ad utilizzare nomi diversi dai numeri per indicare i processori. Si introdussero memorie cache, ed all'inizio degli anni 2000 si introdussero 
diverse forme di parallelismo fisico, e non solo, mantenendo le frequenze inferiori ai 4 GHz. Per molti anni si utilizzava lo stesso processore anche per la gestione 
dello schermo, ma già da parecchi anni è la norma utilizzare due processori separati. 
%% core i3, i5, i7, i9, xeon
Tutti i processori moderni della stessa famiglia sono compatibili con lo stesso linguaggio macchina. La denominazione di un processore indica le sue prestazioni, e sono 
quindi destinati a diversi settori di mercato, per cui l'evoluzione non dipende più dal nome del processore, ma dalla generazione dei processori. Attualmente ci si trova 
in una generazione intermedia tra la tredicesima e la quattordicesima, in generale un salto generazionale viene definito dal livello di integrazione, i processori moderni 
hanno un livello di integrazione di 7 nanometri. 
Sono tutte architetture che presentano fino ad otto core, ma recentemente invece di utilizzare core identici sullo stesso processore sono state introdotte architetture ibride 
i cui core sono di almeno due forme diverse chiamati ``p-core'', per le prestazioni in termini di calcoli complessi, e gli ``e-core'', sono più efficienti in termini di consumo 
di energia. POssono avere fino a 24 stadi di pipeline, che permettono di avere altre forme di parallelismo, senza utilizzare parallelismi fisici. 

Utilizzando un'unica catena di produzione, in base alla qualità in cui vengono prodotti si ottengono diversi processori, disattivando le componenti che non funzionano 
correttamente sul processore, e si vendono quindi come dei processori aventi prestazioni minori rispetto ad una versione completamente funzionante. 

Il processore Intel Core i7 presenta sei core abilitati su otto core disponibili, la sua versione completamente abilitata corrisponde al processore Xeon. Presenta poco più di 
un miliardo di transistor ed un livello di integrazione nell'ordine dei 22 nanometri. 

%% ppt. 

%%\subsection{Architetture ARM}

La società Acorn inventò negli anni '80 questo tipo di architettura bastata sui principi RISC (Acorn RISC Machine). Venne usato sui primi tablet prodotti dalla Apple. Nasce come 
un processore integrato ed a basso consumo energetico. Presenta un modello di commercializzazione diverso rispetto al resto del mercato, utilizzano un'architettura aperta, che 
permette a diverse case produttrici di realizzare questi processori, vendendo le licenze per poter produrre il processore. Ogni processore ARM viene quindi accompagnato dal nome 
dell'azienda che l'ha realizzato. 

Si analizzerà in seguito l'Nvidia Tegra 2, un SOC ``System Of a Chip'' contenente due processori della famiglia ARM, una piccola GPU ed ulteriori componenti.  

%%\subsection{Architettura AVR}
L'architettura AVR corrisponde a processori progettati per elettro-domestici, e per funzioni specifiche. Nacque da un progetto universitario del NIT nel 1996, dal nome dei 
suoi creatori Alf and Vergard RISC Processor. Presenta lo stesso pinout dell'8051 Intel. Presenta vari timer, un orologio interno, trasmettitore di impulso, interfaccia 
di sensori, convertitori analogico-digitali, transponder e comparatore di tensioni. Presenta memorie nell'ordine delle centinaia dei kilobyte per la memoria persistente Flash, 
una memoria programmabile da poche migliaia di byte ``EEPROM'', ed una memoria principale fino ad un massimo di 16 KB, per i microcontrollori più potenti. 

\clearpage

\section{Sistemi di Numerazione Binaria}

%% unità di misura (ppt.)

Quando si misura la memoria si utilizza l'unità di misura byte, corrispondente a 8 bit, e si tende ad utilizzare potenze di due invece di potenze di dieci, utilizzando 
i bit quando si considera una velocità. 

Si utilizza questa notazione per la struttura fisica della memoria, la cua dimensione viene definita dal numero di bit di un indirizzo binario. Dato un indirizzo definito da 
$n$ bit, sono possibili $2^n$ indirizzi distinti di memoria. Per cui è più semplice lavorare con potenze di due quando si analizza la memoria. 

Nei sistemi di numerazione binaria è presente una differenza sostanziale tra il concetto di numero ed il concetto di numerale, un numero è un entità astratta, mentre un 
numerale è una sua possibile rappresentazione in un dato sistema di numerazione. 
Nell'ambito dei calcolatori elettronici il numero di caratteri diversi per poter rappresentare un numero è finito, poiché lo è la dimensione dei registri di memoria. Per cui 
i numeri possono essere rappresentati a precisione finita, e si perdono alcune proprietà come la chiusura rispetto alle sue operazioni, sono presenti errori di arrotondamento di 
un numero. Inoltre quando si vuole rappresentare numeri reali non si possono rappresentare tutti i numeri reali, per cui possono essere rappresentati solo numeri con un numero 
finito di cifre decimali in un dato intervallo. 

\end{document}