\documentclass{article}

\usepackage{cancel}
%\usepackage{tikz}
%\usepackage{circuitikz}
\usepackage{amsmath}
\usepackage[includehead,nomarginpar]{geometry}
\usepackage{graphicx}
\usepackage{amsfonts} 
\usepackage{verbatim}
\usepackage{mathrsfs}  
\usepackage{lmodern}
\usepackage{braket}
\usepackage{bookmark}
\usepackage{fancyhdr}
\usepackage{romanbarpagenumber}
%\usepackage{minted}
%\usepackage{subfig}
\usepackage[italian]{babel}
%\usepackage{float}
\allowdisplaybreaks

\setlength{\headheight}{12.0pt}
\addtolength{\topmargin}{-12.0pt}

\hypersetup{
    colorlinks=true,
    linkcolor=black,
}

\makeatother

\numberwithin{equation}{subsection}

\fancypagestyle{link}{\fancyhf{}\renewcommand{\headrulewidth}{0pt}\fancyfoot[C]{Sorgente del file LaTeX disponibile al seguente link: \url{https://github.com/00Darxk/Calcolatori-Elettronici}}}

\begin{document}

\title{%
    \textbf{Calcolatori Elettronici}  \\ 
    \large Appunti delle Lezioni di Calcolatori Elettronici \\
    \textit{Anno Accademico: 2023/24}}
\author{\textit{Giacomo Sturm}}
\date{\textit{Dipartimento di Ingegneria Civile, Informatica e delle Tecnologie Aeronautiche \\
Università degli Studi ``Roma Tre"}}

\maketitle
\thispagestyle{link}

\clearpage


\pagestyle{fancy}
\fancyhead{}\fancyfoot{}
\fancyhead[C]{\textit{Calcolatori Elettronici - Università degli Studi ``Roma Tre"}}
\fancyfoot[C]{\thepage}
\pagenumbering{Roman}

\tableofcontents

\clearpage
\pagenumbering{arabic}


% prof:
%% big data
%% LLM
%% Tirocini (Azienda/Laboratori) 
%% Incubare Startup

% scopi del corso
%% 6 CFU
%% rec 80-85% programma anni precedenti
%% cenni di assembly alla fine del corso
%% simulatori e casi di studio


%% programma diviso in due parti: (da aggiungere a README)
% I parte:
%% storia e topologia dei calcolatori: calcolatori moderni; x86, ARM, AVL;
%% sistemi di numerazione binaria: rappresentazione floating point numbers, standard IEEE 754;
%% ppt...
% II parte:
%% bus: ppt...
%% micro-architettura di una cpu: ppt...
%% programmazione assembly x86


% modalità d'esame (ppt):


%\section{Introduzione}

\section{Storia e Topologia dei Calcolatori}

%% analisi degli aspetti hardware 
% funzionamento di un microprocessore, facendo riferimento a processori moderni e reali. 

\subsection{Evoluzione delle Architetture}

%% ppt


Un calcolatore è un oggetto che fornisce un risultato, dato un insieme dei dati inseriti. 


L'evoluzione delle architetture dei controllori elettronici si è svolta principalmente negli ultimi settant'anni. Il processo complessivo che ha portato alla nascita 
dei calcolatori moderni viene divisa in generazioni. 
Si chiama generazione zero, l'insieme di calcolatori analogici, progettati per risolvere semplici operazioni, ideati da Pascal e Leibniz. 

Il primo calcolatore programmabile venne ideato da Charles Babbage. Costruì prima una macchina differenziale in grado di calcolare funzioni polinomiali, mentre progettò la 
prima macchina programmabile, completamente analogica, in grado di leggere un input scritto su piastre di rame, e fornire un output, sempre su piastre di rame, utilizzando 
i dati e le operazioni inserite in input. Per poter operare su questi dati di input per fornire istruzioni alla macchina è necessario un linguaggio di programmazione, e la 
prima persona che ha tentato di implementare il linguaggio di Babbage fu Ada. %cognome 


Il passaggio seguente in questa evoluzione venne trainato principalmente dai fondi bellici per realizzare macchine elettromeccaniche. La prima generazione si indica il 
periodo dove vennero realizzati calcolatori abbandonando componenti meccanici. La prima macchina del genere venne create da Alan Turing per decifrare il codice Enigma, 
realizzata tramite valvole, chiamata Colossus. 



Dopo la guerra non servirono più a scopi bellici, per cui si tentò di vendere calcolatori sul mercato, creando la prima società di sviluppo e vendita di calcolatori 
sul mercato, Eniac. 

Negli anni '50 John von Neumann descrisse l'idea di un calcolatore moderno, dove i dati vengono memorizzati su degli indirizzi di memoria. 

L'IBM cominciò la sua storia vendendo calcolatori nel 1953, e continuo ad essere rilevante in questo ambito fino agli anni '80. 
In queste macchine ogni elemento viene definito dal termine ``word'', composto da un certo numero di bit. 


La fine degli anni '50 e l'inizio degli anni '60 vide l'avvento dei transistor, utilizzati in questi anni per la creazione di calcolatori basati su transistor, la società 
rivale della IBM che venne creata in questo periodo fu la DEC. Utilizzando transistor vennero diminuiti i costi, e venne introdotta l'idea di utilizzare uno schermo grafico 
per interagire con l'utente. Il primo calcolatore costruito dalla DEC, PDP-1, fu il primo calcolatore di massa. 

Per accedere ad un calcolatore si utilizzavano dei terminali, tramite un canale di comunicazione chiamato bus, per permettere anche la comunicazione tra elementi interni al 
calcolatore prodotti tra società differenti. 
Si parla comunque di grossi calcolatori per applicazioni scientifiche, militari o di pubblica amministrazione, chiamati ``mainframe'', dove ogni utente accedeva al calcolatore 
tramite terminali. 

Fino agli anni '80 non vennero introdotti cambiamenti radicali all'architettura dei calcolatori, invece i miglioramenti di questi periodi ai calcolatori riguardarono soprattutto 
l'ottimizzazione del software e dell'hardware. 

I primi ``Personal Computer'' vennero introdotti negli anni '80, dall'IBM, che fornì pubblicamente l'architettura del calcolatore. In seguito aumentò in enorme maniera 
l'utilizzo di PC, trainato dall'aumento delle capacità della CPU, e dalla diminuzione dei costi delle memorie principali e secondarie. 


%% ppt... 

% V Computer Invisibili
La maggior parte dei dispositivi moderni contengono microcontrollori, piccoli processori, distribuiti in un contesto completamente pervasivo, su ogni dispositivo collegato 
ad una qualche fonte di energia. Questo concetto viene chiamato anche dell'``Internet of Things''.  





\clearpage










\end{document}