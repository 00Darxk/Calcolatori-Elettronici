\documentclass{standalone}
\begin{document}
    \begin{tabular}{|c||c|c|c|c|}
        \hline
        Nome & ID & Operazione & Argomenti & Valore Restituito\\
        \hline\hline
        \verb|_EXIT|&1&Interrompe un processo&0: successo/1: errore&\\
        \hline
        \verb|_READ|&3&Legge n byte da fd, in buf& fd, buf, n & Numero di byte letti\\
        \hline
        \verb|_WRITE|&4&Scrive n byte su fd, da buf& fd, buf, n& Numero di byte scritti\\
        \hline
        \verb|_OPEN|&5& Apre un file in lettura/scrittura & nome, 0: rd/1: wr/2: rd-wr & Descrittore di file (fd)\\
        \hline
        \verb|_CLOSE|&6&Chiude un file aperto& fd& 0, se ha successo \\
        \hline 
        \verb|_CREATE|& 8 & Crea un nuovo file & nome, permessi UNIX & Descrittore di file (fd)\\
        \hline
        \verb|_LSEEK| &19& Sposta fd di offset bytes & fd, offset, 0/1/2& Nuovo fd\\
        \hline
        \verb|_GETCHAR| &117& Legge un carattere in input && Carattere letto, in \verb|AL|\\
        \hline
        \verb|_SPRINTF|&121& Stampa stringa formattata su buf&buf, stringa, args.&\\
        \hline
        \verb|_PUTCHAR|&122&Scrive un carattere in output &Carattere&\\
        \hline
        \verb|_SSCANF|&125&Legge argomenti da buf&buf, stringa, args.&\\
        \hline
        \verb|_PRINTF|&127&Stampa una stringa formattata in output&stringa, args.&\\
        \hline
    \end{tabular}
\end{document}