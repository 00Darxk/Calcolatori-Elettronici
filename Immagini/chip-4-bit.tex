\documentclass{standalone}
\usepackage{circuitikz}
\tikzset{registro/.style={flipflop,
flipflop def={t1 = D, t6=Q, c3=1}
}}
\begin{document}
\begin{circuitikz}
    \draw
    (0,0)node[above]{CK}to[short,d-]++(0,-2)
    to[short,-*]++(1,0)coordinate(ff1)
    to[short,-*]++(3,0)coordinate(ff2)
    to[short,-*]++(3,0)coordinate(ff3)
    to[short]++(3,0)coordinate(ff4)
    (ff1)to[short]++(0,1)node[registro, anchor=pin 3](a){}
    (a.pin 1)to[short]++(0,0.25)node[above]{I$_0$}
    (a.pin 6)to[short]++(0,0.25)node[above]{O$_0$}
    (a.south)to[short]++(0,-1)
    to[short,-*]++(3,0)coordinate(t)
    (t)to[short]++(0,1)
    (t)to[short,-*]++(3,0)coordinate(t)
    (t)to[short]++(0,1)
    (t)to[short,-*]++(3,0)coordinate(t)
    (t)to[short]++(0,1)
    (t)to[short]++(2,0)
    to[short,-d]++(0,2)node[above]{CLR}
    (ff2)to[short]++(0,1)node[registro, anchor=pin 3](a){}
    (a.pin 1)to[short]++(0,0.25)node[above]{I$_1$}
    (a.pin 6)to[short]++(0,0.25)node[above]{O$_1$}
    (ff3)to[short]++(0,1)node[registro, anchor=pin 3](a){}
    (a.pin 1)to[short]++(0,0.25)node[above]{I$_2$}
    (a.pin 6)to[short]++(0,0.25)node[above]{O$_2$}
    (ff4)to[short]++(0,1)node[registro, anchor=pin 3](a){}
    (a.pin 1)to[short]++(0,0.25)node[above]{I$_3$}
    (a.pin 6)to[short]++(0,0.25)node[above]{O$_3$}

    ;
\end{circuitikz}
\end{document}