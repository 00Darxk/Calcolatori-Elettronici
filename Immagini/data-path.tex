\documentclass{standalone}
\usepackage{circuitikz}
\usepackage{rotating}
\tikzset{ALU/.style={muxdemux,
    muxdemux def={Lh=5, NL=2, Rh=2, NR=1, NB=1, NT=2, w=2,
    inset w=1, inset Lh=2, inset Rh=0, square pins=1}},
    REG/.style={muxdemux,
    muxdemux def={Lh=1, Rh=1, NL=1, NB=0, NR=1, w=5}},
    MBR/.style={muxdemux,
    muxdemux def={Lh=1, Rh=1, NL=1, NB=0, NR=2, w=5}},
    MAR/.style={muxdemux,
    muxdemux def={Lh=1, Rh=1, NL=2, NB=0, NR=0, w=5}},
    MDR/.style={muxdemux,
    muxdemux def={Lh=1, Rh=1, NL=2, NB=0, NR=1, w=5}},
    H/.style={muxdemux,
    muxdemux def={Lh=1, Rh=1, NL=1, NB=0, NR=0, w=5}},
    SH/.style={muxdemux,
    muxdemux def={Lh=1, Rh=1, NL=0, NB=1, NR=1, w=5}}}
\begin{document}
\begin{circuitikz}
    \draw
    (0,0)coordinate(o)node[MAR](mar){MAR}
    (o)++(0,-1)coordinate(o)node[MDR](mdr){MDR}
    (o)++(0,-1)coordinate(o)node[MDR](pc){PC}
    (o)++(0,-1)coordinate(o)node[MBR](mbr){MBR}
    (o)++(0,-1)coordinate(o)node[REG](sp){SP}
    (o)++(0,-1)coordinate(o)node[REG](lv){LV}
    (o)++(0,-1)coordinate(o)node[REG](cpp){CPP}
    (o)++(0,-1)coordinate(o)node[REG](tos){TOS}
    (o)++(0,-1)coordinate(o)node[REG](opc){OPC}
    (o)++(0,-1)coordinate(o)node[H](h){H}

    (h.south)to[short,i=\mbox{}]++(0,-0.25)node[ALU, anchor=lpin 2, rotate=270](alu){\rotatebox{90}{ALU}}

    (mdr.rpin 1)to[short]++(0.3,0)to[short,i=\mbox{}](alu.lpin 1)
    (pc.rpin 1)to[short,i=\mbox{}]++(0.3,0)
    (mbr.rpin 1)to[short,i=\mbox{}]++(0.3,0)
    (mbr.rpin 2)to[short,i=\mbox{}]++(0.3,0)
    (sp.rpin 1)to[short,i=\mbox{}]++(0.3,0)
    (lv.rpin 1)to[short,i=\mbox{}]++(0.3,0)
    (cpp.rpin 1)to[short,i=\mbox{}]++(0.3,0)
    (tos.rpin 1)to[short,i=\mbox{}]++(0.3,0)
    (opc.rpin 1)to[short,i=\mbox{}]++(0.3,0)

    (alu.tpin 1)to[short,i=\mbox{},-d]++(0.5,0)node[right]{N}
    (alu.tpin 2)to[short,i=\mbox{},-d]++(0.5,0)node[right]{Z}
    (alu.bpin 1)++(-0.5,0)to[multiwire=6,i=\mbox{},d-](alu.bpin 1)
    (alu.rpin 1)node[SH, anchor=north](sh){Shifter}
    (sh.rpin 1)++(0.5,0)to[multiwire=2,i=\mbox{},d-](sh.rpin 1)

    (sh.bpin 1)to[short,i=\mbox{}]++(-2.65,0)node[above right]{Bus C}
    to[short,i=\mbox{},-*](h.lpin 1)
    to[short,-*](opc.lpin 1)
    to[short,-*](tos.lpin 1)
    to[short,-*](cpp.lpin 1)
    to[short,-*](lv.lpin 1)
    to[short,-*](sp.lpin 1)
    to[short,-*](pc.lpin 1)
    to[short,-*](mdr.lpin 1)
    to[short](mar.lpin 1)
    
    node[above right]at(alu.blpin 2){A}
    node[above right]at(alu.blpin 1){Bus B}

    ;
    \draw[<-, thick](mar.center)++(-1,-0.25)--++(0,-0.2);
    \draw[<-, thick](mdr.center)++(-1,-0.25)--++(0,-0.2);
    \draw[<-, thick, gray](mdr.center)++(1,-0.25)--++(0,-0.2);
    \draw[<-, thick](pc.center)++(-1,-0.25)--++(0,-0.2);
    \draw[<-, thick, gray](pc.center)++(1,-0.25)--++(0,-0.2);
    \draw[<-, thick, gray](mbr.center)++(1.25,-0.25)--++(0,-0.2);
    \draw[<-, thick, gray](mbr.center)++(1,-0.25)--++(0,-0.2);
    \draw[<-, thick](sp.center)++(-1,-0.25)--++(0,-0.2);
    \draw[<-, thick, gray](sp.center)++(1,-0.25)--++(0,-0.2);
    \draw[<-, thick](lv.center)++(-1,-0.25)--++(0,-0.2);
    \draw[<-, thick, gray](lv.center)++(1,-0.25)--++(0,-0.2);
    \draw[<-, thick](cpp.center)++(-1,-0.25)--++(0,-0.2);
    \draw[<-, thick, gray](cpp.center)++(1,-0.25)--++(0,-0.2);
    \draw[<-, thick](tos.center)++(-1,-0.25)--++(0,-0.2);
    \draw[<-, thick, gray](tos.center)++(1,-0.25)--++(0,-0.2);
    \draw[<-, thick](opc.center)++(-1,-0.25)--++(0,-0.2);
    \draw[<-, thick, gray](opc.center)++(1,-0.25)--++(0,-0.2);
    \draw[<-, thick](h.center)++(-1,-0.25)--++(0,-0.2);

    \draw[->, very thick, gray](mar.blpin 2)--++(-1.5,0);
    \draw[<->, very thick, gray](mdr.blpin 2)--++(-1.5,0);
    \draw[->, very thick, gray](pc.blpin 2)--++(-1.5,0);
    \draw[<-, very thick, gray](mbr.blpin 1)--++(-1.5,0);

    \draw[dashed, thick](2.5,0.5)--(-3,0.5)--(-3,-3.5)node[midway, left]{RAM}--(2.5,-3.5)--(2.5,0.5);

    \node[rectangle, very thick, draw,anchor=0, minimum width=0.5cm, minimum height=0.5cm]at(mbr.east){};

\end{circuitikz}
\end{document}